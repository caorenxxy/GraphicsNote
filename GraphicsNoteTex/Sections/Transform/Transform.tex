\section{Transform}
\subsection{Projection}
投影是指从camera space($\Omega_{cam}$)到NDC space($\Omega_{NDC}$)的变换,其中NDC space的$x,y,z$的范围都是$[-1,1]$。假设某个点在camera space里的坐标为 $(x,y,x)$,我们把这个点投影到film上$(x_p,y_p,z_p)$,然后再经过normalization,得到NDC space中的坐标$(x_N, y_N,z_N)$。定义film到camera的距离为$filmDis$,根据$fovy$,$filmDis$以及$aspect$,可以计算出film在camera space里的横坐标范围$[l,r]$,纵坐标范围$[b,t]$。定义投影矩阵$M_{proj}: \Omega_{cam}\to \Omega_{NDC}$,$\bold{x} = [x, y, z, 1]^T$,$\bold{x}_p = [x_p, y_p, z_p, 1]^T$,$\bold{x}_N = [x_N, y_N, z_N, 1]^T$ (这里需要了解齐次坐标的知识)。由投影矩阵的定义可得$\bold{x}_N = M_{proj}\bold{x}$。并且这里假设近裁剪端和远裁剪端到camera的距离分别为$n,f$
\subsubsection{Orthographic}
\subsubsection{Perspective}
由几何关系
\begin{displaymath}
\frac{x_p}{x} = \frac{y_p}{y} = \frac{filmDis}{z}
\end{displaymath}

可以得到
\begin{gather*}
x_p = \frac{filmDis}{z} \cdot x\\
y_p = \frac{filmDis}{z} \cdot y
\end{gather*}

对于$x_p$我们可以得到以下推导
\begin{gather*}
l \leq x_p \leq r\\
0 \leq x_p - l \leq r-l\\
0 \leq \frac{x_p - l}{r - l} \leq 1\\
0 \leq 2\frac{x_p - l}{r - l} \leq 2\\
-1 \leq \frac{2 x_p}{r-l} - \frac{r + l}{r - l} \leq 1
\end{gather*}

同理也可以得到
\begin{displaymath}
-1 \leq \frac{2y_p}{t - b} - \frac{t + b}{t - b} \leq 1
\end{displaymath}
\\

总结一下,我们可以的得到关于$x, y$的投影后的关系
\begin{displaymath}
\left\{
\begin{gathered}
-1 \leq \frac{2}{r - l}\cdot\frac{filmDis}{z}\cdot x - \frac{r + l}{r - l} \leq 1\\
-1 \leq \frac{2}{t - b}\cdot\frac{filmDis}{z}\cdot y - \frac{t + b}{t - b} \leq 1
\end{gathered}
\right.
\end{displaymath}
\\

由此我们可以构造出一部分投影矩阵的元素
\begin{displaymath}
M_{proj} = 
\begin{bmatrix}
\frac{2filmDis}{r - l} & 0  & -\frac{r + l}{r - l} & 0\\
0 & \frac{2filmDis}{t - b} & -\frac{t + b}{t - b} & 0\\
0 & 0 & A & B\\
0 & 0 & 1 & 0
\end{bmatrix}
\end{displaymath}
其中$A, B$为一个待定的元素,注意这里的投影矩阵其实并不是唯一的,还有其他的构造方法。

我们写出从camera space到NDC space的表达,然后再进行normalization:
\begin{displaymath}
\begin{bmatrix}
x_N'\\
y_N'\\
z_N'\\
w_N
\end{bmatrix} = 
\begin{bmatrix}
\frac{2filmDis}{r - l} & 0 & -\frac{r + l}{r - l} & 0\\
0 & \frac{2filmDis}{t - b} & -\frac{t + b}{t - b} & 0\\
0 & 0 & A & B\\
0 & 0 & 1 & 0
\end{bmatrix}
\begin{bmatrix}
x\\
y\\
z\\
1
\end{bmatrix} =
\left[\begin{gathered}
\frac{2filmDis}{r - l}\cdot x - \frac{r + l}{r - l} \cdot z\\
\frac{2filmDis}{t - b} \cdot y - \frac{t + b}{t - b}\cdot z\\
zA + B\\
z
\end{gathered}\right]\Leftrightarrow
\end{displaymath}

\begin{displaymath}
\begin{bmatrix}
x_N\\
y_N\\
z_N\\
1
\end{bmatrix} = 
\begin{bmatrix}
x_N'/w_N\\
y_N'/w_N\\
z_N'/w_N\\
1
\end{bmatrix} = 
\begin{bmatrix}
\frac{2x}{r - l}\cdot\frac{1}{z} - \frac{r + l}{r - l}\\
\frac{2y}{t - b}\cdot\frac{1}{z} - \frac{t + b}{t - b}\\
(zA + B) / z \\
1
\end{bmatrix}
\end{displaymath}

由裁剪锥体的性质可得,在$n$处坐标为被映射为$-1$,在$f$处坐标映射成$1$
\begin{displaymath}
\left\{
\begin{aligned}{}
\frac{nA + B}{n} & = & -1\\
\frac{fA + B}{f} & = & 1
\end{aligned}
\right.
\Rightarrow
\left\{
\begin{aligned}
&A = \frac{f + n}{f - n}\\
&B = \frac{2nf}{f - n}
\end{aligned}
\right.
\end{displaymath}
注意:这个结果和pbrt里的有所不同,原因是pbrt中的范围是$[0,1]$
\\

最后得到完整的投影矩阵:
\begin{displaymath}
M_{proj} = 
\begin{bmatrix}
\frac{2filmDis}{r - l} & 0 & -\frac{r + l}{r - l} & 0\\
0 & \frac{2filmDis}{t - b} & -\frac{t + b}{t - b} & 0\\
0 & 0 & \frac{f + n}{f - n} & \frac{2nf}{f - n}\\
0 & 0 & 1 & 0
\end{bmatrix}
\end{displaymath}
\\

一般情况下$r = -l$,$t = -b$,因此投影矩阵可以简化成:
\begin{displaymath}
M_{proj} = 
\begin{bmatrix}
\frac{1}{r} & 0 & 0 & 0\\
0 & \frac{1}{t} & 0 & 0\\
0 & 0 & \frac{f + n}{f - n} & \frac{2nf}{f - n}\\
0 & 0 & 1 & 0
\end{bmatrix}
\end{displaymath}